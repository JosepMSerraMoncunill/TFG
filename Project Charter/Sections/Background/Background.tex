\section{Background}
It is well known that transfer of heat and mass, fluid flow, chemical reactions, and other related processes play a vital role in the world. These processes can be observed in an infinite number of situations. Almost all methods of power production require fluid flow and heat transfer. The same can be said about heating and cooling in buildings. Thermo-fluid processes are present in ovens, furnaces, heat exchangers, reactors and condensers. Aircrafts and automobiles require a fluid flow and heat exchange in order to function. In electrical and electronic components, heat transfer is a very important issue. Storms and fires are also caused by heat and mass transfer. Even the human body is under a process of continuous flow of heat and mass.\\
\\
Knowing the great importance of these factors in the world, it is no wonder that an accurate understanding of these processes is needed, as well as an efficient way to predict them. However, although this processes can be fully understood and expressed in a set of mathematical equations, the so called conservation equations, the nature of this processes often leads to very non-linear coupled equations and a general solution is impossible to obtain. These equations are: conservation of mass (one time dependant equation), conservation of momentum (three time dependant equations, one for each direction), and conservation of energy (one time dependant equation), often called Navier-Stokes equations. At this point, two options are possible in order to obtain the solution of these equations for a given problem: experimental tests or theoretical resolution.\\
\\
Experimental investigation using full-scale tests allow scientists con predict how identical processes will react to the same characteristics. However, these kind of tests are very expensive and, therefore, small-scale models must be used. In the process of going from a full-scale model to a small-scale one, some characteristics of the full-scale problem must be extrapolated, and the same happens with the results obtained in the small-scale one. Additionally, some features must be omitted. It should also be taken into account that results depend of the instrument precision and the discrete number of sensors used. These problems reduce the usefulness of experimental results.\\
\\
Theoretical resolution implies the resolution in all the domain of a system of non-linear coupled equations, which nowadays has resulted to be impossible. Therefore, simplifications must be made in order to obtain a solution, such as incompressibility or steady conditions. Even with simplifications, only a small fraction of problems can be solved, and with further simplifications, the more problems can be solver, but the less exact becomes the solution. In some cases, the least simplified solutions that is able to solve the problem is too simplified to obtain an exact solution.\\
\\
At this point is where numerical methods become relevant. The basics of computational fluid dynamics (CFD) consist in discretizing the domain in a finite number of zones. Inside these regions, conservation equations are linearized. This turns a system of a few non-linear equations impossible to solve into a large system of linear equations that can be solved. Although this new system of equations gives an approximated solution, with enough discretization is possible to obtain a solution with enough accuracy. However, it must be noted that since the size of the discretization depends of the performance of the computer, it has not been until recent years that computers have become powerful enough to give accurate solutions. Nevertheless, in its current state, CFD provide some advantages compared with other types of resolution, as stated in \cite{Patankar}\\
\\
\begin{itemize}
\item \textbf{Low cost} The most important factor in the resolution of a problem is its cost. And in CFD, in most applications, it is much less expensive to run the solution in a computer than to perform an experimental test. This is more noticeable as problems get more complicated.
\item \textbf{Speed} Compared with an experimental test, a computational test takes less time to be implemented and to evaluate all the options in order to obtain an optimum design.
\item \textbf{Complete information} Computer resolutions give information about all relevant variables in the entire domain, unlike in experimental tests, where only data from a few points can be obtained and the presence of the sensors can cause disturbances in the flow.
\item \textbf{Ability to simulate realistic conditions} For a computer program, solving a full-scale problem without simplifications in not a difficult task, since full-dimensions, reactivity and fast processes can easily be computed.
\item \textbf{Ability to simulate ideal conditions} In the study of a basic phenomenon, only a few essential parameters are important, and the others are omitted. This is difficult to accomplish in experimental tests, but very easy to implement in computational resolution.
\end{itemize}
However, although the advantages are very remarkable, the use of computational calculations also has its disadvantages, as \cite{Patankar} also says. The main one being that a computational resolution is based on a mathematical method, unlike an experimental test, which observes reality itself. Therefore, no matter how good the numerical method is, if the mathematical method is incorrect, the computational resolution is incorrect as well.\\
\\
In those cases where the analytical solution exists (steady heat transfer, for example), computational resolution of the problem becomes irrelevant compared to the analytical resolution. Additionally, in problems with high complexity (such as very irregular geometry, strong non-linearity, etc.) the cost of the computational solution may be superior to a small-scale test. And in addition to that, phenomena such as turbulence, which are very fast and happen in a very small scale require a computational cost extremely high that makes computational resolution of the problem very unpractical. Moreover, when does not exist an analytical solution of the problem, it is not certain if the results will adjust with reality or not.\\
\\
Nevertheless, in the majority of cases, the advantages of computational resolution greatly surpass the disadvantages, especially when the mathematical model in which is based is valid in the entire domain. With its low cost and fast performance, computational methods provide a very robust resolution of the problem without having to use simplified mathematical models or experimental tests.
\pagebreak